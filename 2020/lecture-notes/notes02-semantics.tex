\documentclass[11pt]{article}
\usepackage{../../tex/math-cmds}
\usepackage{../../tex/analysis}
\usepackage{IEEEtrantools}

\title{Lecture Notes: Program Semantics}
\author{17-355/17-665/17-819: Program Analysis (Spring 2020)\\
        Claire Le Goues\footnote{These notes were developed together with Jonathan Aldrich}\\
		{\tt clegoues@cs.cmu.edu}}
\date{}
\begin{document}

\maketitle


\section{Operational Semantics} 

To reason about analysis correctness, we need a clear definition of what a
program \textit{means}.  One way to do this is using natural language (e.g., the
Java Language Specification).  However, although natural language specifications
are accessible, they are also often imprecise. This can lead to many problems,
including incorrect compiler implementations or program analyses.

A better alternative is a formal definition of program semantics.  We begin with
\textit{operational semantics}, which mimics, at a high level, the operation of
a computer executing the program.  Such a semantics also reflects the way that
techniques such as dataflow analysis or Hoare Logic reason about the program, so
it is convenient for our purposes.

There are two broad classes of operational semantics: \emph{big-step operational
  semantics}, which specifies the entire operation of a given expression or
statement; and \emph{small-step operational semantics}, which specifies the
operation of the program one step at a time.

\subsection{\WhileLang: Big-step operational semantics}

We'll start by restricting our attention to arithmetic expressions, for
simplicity.  What is the meaning of a \WhileLang expression?  Some expressions, like
a natural number, have a very clear meaning: The ``meaning'' of 5 is just, well,
5.  But what about $x + 5$?  The meaning of this expression clearly depends on
the value of the variable $x$.  We must \emph{abstract} the value of variables
as a function from variable names to integer values:

\begin{equation*}
E \in \textit{Var} \rightarrow \Integer 
\end{equation*}

%Note(JA): I took out \mathcal{E} (distinct from $E$) as this seems non-standard (most PL theory uses the same symbol for the domain and the metavariable) and potentially confusing to students.  Plus we weren't using \mathcal{E} consistently anywhere else.

\noindent Here $E$ denotes a particular program
\emph{state}.  The meaning of an expression with a variable like $x + 5$
involves ``looking up'' the $x$'s value in the associated $E$, and substituting
it in.
%
Given a state, we can write a \emph{judgement} as follows:

\begin{equation*}
\langle a, E \rangle \Downarrow n
\end{equation*}

\noindent This means that given program state $E$, the expression $e$ evaluates to $n$.
This formulation is called \emph{big-step} operational semantics; the
$\Downarrow$ judgement relates an expression and its ``meaning.''\footnote{Note
  that I have chosen $\Downarrow$ because it is a common notational convention;
  it's not otherwise special. This is true for many notational choices in formal
  specification.}
%
We then build up the meaning of more complex expressions using \emph{rules of
  inference} (also called \emph{derivation} or \emph{evaluation} rules).  An
inference rule is made up of a set of judgments above the line, known as
\emph{premises}, and a judgment below the line, known as the \emph{conclusion}.
The meaning of an inference rule is that the conclusion holds if all of the
premises hold:

\begin{equation*}
\infer{conclusion}{premise_1 & premise_2 & \ldots & premise_n}
\end{equation*}

\noindent 
An inference rule with no premises is an \emph{axiom}, which is always
true. For example, integers always evaluate to themselves, and the meaning of a
variable is its stored value in the state:

\begin{center}
\begin{IEEEeqnarraybox}{s?s}
\infer[$\textit{big-int}$]{\langle n, E  \rangle \Downarrow n}{} &
\infer[$\textit{big-var}$]{\langle x, E \rangle \Downarrow E(x)} {} \\
\end{IEEEeqnarraybox}
\end{center}

\noindent Addition expressions illustrate a rule with premises: 

\begin{equation*}
\infer[\textit{big-add}]{\langle a_1 + a_2, E  \rangle \Downarrow n_1 + n_2}{\langle a_1, E \rangle \Downarrow
  n_1& \langle a_2, E \rangle \Downarrow n_2}
\end{equation*}

But, how does the value of $x$ come to be ``stored'' in $E$?  For that, we must
consider \WhileLang \emph{Statements}.  Unlike expressions, statements have no
direct result.  However, they can have \emph{side effects}.  That is to say: the
``result'' or \emph{meaning} of a Statement is a \emph{new state}.  The
judgement $\Downarrow$ as applied to statements and states therefore looks like:

\begin{equation*}
\langle S, E \rangle \Downarrow E' 
\end{equation*}

This allows us to write inference rules for statements, bearing in mind that
their \emph{meaning} is not an integer, but a new state.  The meaning of
\texttt{skip}, for example, is an unchanged state:

\begin{equation*}
\infer[\textit{big-skip}]{\langle \mathtt{skip}, E  \rangle \Downarrow E}{}
\end{equation*}

\noindent Statement sequencing, on the other hand, does involve premises:

\[
\infer[\textit{big-seq}]{\langle S_1 ; S_2, E \rangle \Downarrow E''}{\langle S_1,E \rangle \Downarrow E'
 & \langle S_2, E' \rangle \Downarrow E''}
\]

The \texttt{if} statement involves two rules, one for if the boolean
predicate evaluates to \texttt{true} (rules for boolean expressions not shown),
and one for if it evaluates to \texttt{false}.  I'll show you just the first one
for demonstration:

\[
\infer[\textit{big-iftrue}]{\langle\mathtt{if}~P~\mathtt{then}~S_1~\mathtt{else}~S_2, E \rangle \Downarrow
  E'}{\langle P, E \rangle \Downarrow \mathtt{true} & \langle S_1, E \rangle \Downarrow
  E'}
\]

\noindent \emph{What should the second rule for} \texttt{if} \emph{look like?}

\vspace{1em}

\noindent This brings us to assignments, which produce a new state in which the variable being assigned to is mapped to the value from the right-hand side.  We write this with the notation $E[x \mapsto n]$, which can be read ``a new state that is the same as $E$ except that $x$ is mapped to $n$.''

\begin{equation*}
\infer[\textit{big-assign}]{\langle x := a, E \rangle \Downarrow E[x \mapsto n]} {\langle a, E \rangle \Downarrow n} 
\end{equation*}

%Note(JA): I changed := to \mapsto when updating E for consistency with other places in the semantics. Also students are getting confused and thinking this is an imperative update.

Note that the update to the state is modeled \textit{functionally}; the variable $E$ still refers to the old state, while $E[x \mapsto n]$ is the new state represented as a mathematical map.

Fully specifying the semantics of a language requires a judgement rule like this
for every language construct.  These notes only include a subset for \WhileLang, for
brevity.  

\exercise{1} What are the rule(s) for \texttt{while}?

\subsection{\WhileLang: Small-step operational semantics}

Big-step operational semantics has its uses.  Among other nice features, it
directly suggests a simple interpreter implementation for a given language.
However, it is difficult to talk about a statement or program whose evaluation
does not terminate.  Nor does it give us any way to talk about intermediate
states (so modeling multiple threads of control is out).

Sometimes it is instead useful to define a \emph{small-step operational
  semantics}, which specifies program execution one step at a time.  We refer to
the pair of a statement and a state ($\langle S, E \rangle$) as a \emph{configuration}. Whereas
big step semantics specifies program meaning as a function between a
configuration and a new state, small step models it as a step from one
configuration to another.

You can think of small-step semantics as a set of rules that we repeatedly apply
to configurations until we reach a \emph{final configuration} for the language
($\langle \mathtt{skip}, E \rangle$, in this case) if ever.\footnote{Not all statements reach
  a final configuration, like \texttt{while true do skip}.}  We write this new
judgement using a slightly different arrow: $\rightarrow$.  $\langle S, E \rangle \rightarrow
\langle S', E' \rangle$ indicates one step of execution; $\langle S, E \rangle \rightarrow^* \langle S', E' \rangle$
indicates zero or more steps of execution.  We formally define multiple execution steps as follows:

\begin{center}
\begin{IEEEeqnarraybox}{s?s}
\infer[$\textit{multi-reflexive}$]{\langle S, E \rangle \rightarrow^* \langle S, E \rangle}{} &
\infer[$\textit{multi-inductive}$]{\langle S, E \rangle \rightarrow^* \langle S'', E'' \rangle}{\langle S, E \rangle \rightarrow \langle S', E' \rangle & \langle S', E' \rangle \rightarrow^* \langle S'', E'' \rangle} \\
\end{IEEEeqnarraybox}
\end{center}


To be complete, we should also define auxiliary small-step operators
$\rightarrow_a$ and $\rightarrow_b$ for arithmetic and boolean expressions,
respectively; only the operator for statements results in an updated state (as
in big step).  The types of these judgements are thus:

\[
\begin{array}{lll}
\rightarrow & : & (\mathtt{Stmt} \cross E) \rightarrow (\mathtt{Stmt} \cross E) \\
\rightarrow_a & : & (\mathtt{Aexp} \cross E) \rightarrow \mathtt{Aexp} \\
\rightarrow_b & : & (\mathtt{Bexp} \cross E) \rightarrow \mathtt{Bexp} \\
\end{array}
\]

\noindent We can now again write the semantics of a \WhileLang  program as new rules
of inference.  Some rules look very similar to the big-step rules, just with a
different arrow.  For example, consider variables:

\begin{equation*}
\infer[\textit{small-var}]{\langle x, E \rangle \rightarrow_a E(x)}{}
\end{equation*}

\noindent Things get more interesting when we return to statements.  Remember,
small-step semantics express a single execution step.  So, consider an
\texttt{if} statement:

\begin{equation*}
\infer[\textit{small-if-congruence}]{\langle \mathtt{if}~P~\mathtt{then}~S_1~\mathtt{else}~S_2, E \rangle \rightarrow \langle \mathtt{if}~P'~\mathtt{then}~S_1~\mathtt{else}~S_2, E \rangle}
{ \langle P, E \rangle \rightarrow_b~P'}
\end{equation*}


\begin{equation*}
\infer[\textit{small-iftrue}]{\langle \mathtt{if}~\mathtt{true}~\mathtt{then}~S_1~\mathtt{else}~S_2, E \rangle \rightarrow
  \langle S_1,E \rangle}{}
\end{equation*}

\exercise{2} We have again omitted the \textit{small-iffalse} case, as well as rule(s) for
\texttt{while}, as exercises to the reader.\\

\noindent Note also the change for statement sequencing:

\begin{equation*}
\infer[\textit{small-seq-congruence}]{\langle S_1;S_2, E \rangle \rightarrow \langle S_1';S_2, E' \rangle}{
\langle S_1, E \rangle \rightarrow \langle S_1',E' \rangle
}{}
\end{equation*}


\begin{equation*}
\infer[\textit{small-seq}]{\langle \mathtt{skip};S_2, E \rangle \rightarrow
  \langle S_2,E \rangle}{}
\end{equation*}


\subsection{\WhileThAddr: Small-step semantics}

The ideas behind big- and small-step operational semantics are consistent
across languages, but the way they are written can vary based on what is
notationally convenient for a particular language or analysis.  
\WhileThAddr is slightly different from \WhileLang, so beyond requiring different rules for its 
different constructs, it makes sense to modify our small-step notation a bit for defining the 
meaning of a \WhileThAddr program.

First, let's revisit the \emph{configuration} to account for the slightly different \emph{meaning} of a
\WhileThAddr program.  As before, the configuration must include the state, which we still call $E$, mapping 
variables to values. However, a well-formed, terminating \WhileLang program
was effectively a single statement that can be iteratively reduced to
\texttt{skip}; a \WhileThAddr program, on the other hand, is a mapping from
natural numbers to program instructions. 
So, instead of a statement that is being reduced in steps, the \WhileThAddr $c$ must
includes a program counter $n$, representing the next instruction to be
executed.  

Thus, a configuration $c$ of the abstract machine for \WhileThAddr
must include the stored program $P$ (which we will generally treat implicitly),
the state environment $E$, and the current program counter $n$
representing the next instruction to be executed ($c \in E \cross \Natural$). 
The abstract machine executes one step at a time, executing the instruction that
the program counter points to, and updating the program counter and environment
according to the semantics of that instruction.  

This adds a tiny bit of complexity to the inference rules, because
they must explicitly consider the mapping between line number/labels and program
instructions.  We represent execution of
the abstract machine via a judgment of the form $P \vdash \langle E,n \rangle
\leadsto \langle E',n' \rangle$ The judgment reads: ``When executing the program $P$,
executing instruction $n$ in the state $E$ steps to a new state $E'$
and program counter $n'$.''\footnote{I could have used the same $\rightarrow$ I
  did above instead of $\leadsto$, but I don't want you to mix them up.}
To see this in action, consider a simple inference rule defining the semantics of the constant
 assignment instruction:

\[
\infer[\textit{step-const}]{P \vdash \langle E,n \rangle \leadsto \langle E[x \mapsto m],n+1 \rangle}{P[n] = x := m}
\]

This states that in the case where the $n$th instruction of the program P (looked up using $P[n]$) is a constant assignment $x := m$, the abstract machine takes a step to a state in which the state $E$ is updated to map $x$ to the constant $m$, written as $E[x \mapsto m]$, and the program counter now points to the instruction at the following address $n+1$.
%
We similarly define the remaining rules:

\[
\begin{array}{c}
\infer[\textit{step-copy}]
	{P \vdash \langle E,n \rangle \leadsto \langle E[x \mapsto E[y]],n+1 \rangle}
	{P[n] = x := y}\\[3ex]
    
\infer[\textit{step-arith}]
	{P \vdash \langle E, n \rangle \leadsto \langle E[x \mapsto m],n+1 \rangle}
	{P[n] = x := y ~op~ z & E[y] ~\mathbf{op}~ E[z] = m}\\
\end{array}
\]
\[
\begin{array}{c}	

\infer[\textit{step-goto}]
	{P \vdash \langle E,n \rangle \leadsto \langle E,m \rangle}
	{P[n] = \mbox{goto}~m}\\[3ex]
	
\infer[\textit{step-iftrue}]
	{P \vdash \langle E,n \rangle \leadsto \langle E,m \rangle}
	{P[n] = \mbox{if}~x~op_r~0~\mbox{goto}~m & E[x] ~\mathbf{op_r}~ 0 = true}\\[3ex]
	
\infer[\textit{step-iffalse}]
	{P \vdash \langle E,n \rangle \leadsto \langle E,n+1 \rangle}
	{P[n] = \mbox{if}~x~op_r~0~\mbox{goto}~m & E[x] ~\mathbf{op_r}~ 0 = false}\\[3ex]
	
\end{array}
\]


\subsection{Derivations and provability} 

Among other things, we can use operational semantics to prove that concrete
program expressions will evaluate to particular values.  We do this by chaining
together rules of inference (which simply list the hypotheses necessary to
arrive at a conclusion) into \emph{derivations}, which interlock instances of
rules of inference to reach particular conclusions.  For example:

\begin{equation*}
\infer{\langle (4 * 2) - 6, E_1 \rangle \Downarrow 2}
{\infer{\langle 4 * 2, E_1 \rangle \Downarrow 8}{\langle 4,E_1 \rangle \Downarrow 4& \langle 2,E_1 \rangle \Downarrow 2}
& \langle 6, E_1 \rangle \Downarrow 6}
\end{equation*}

We say that $\langle a, E \rangle \Downarrow n$ is \emph{provable} (expressed
mathematically as $\vdash \langle a,E \rangle \Downarrow n$) if there exists a
well-formed derivation with $\langle a, E \rangle \Downarrow n$ as its
conclusion.  ``Well formed'' simply means that every step in the derivation is a
valid instance of one of the rules of inference for this system.

A proof system like our operational semantics is \emph{complete} if every true
statement is provable.  It is \emph{sound} (or \emph{consistent}) if every
provable judgement is true.
%
%Typically, a system of semantics is always complete, unless you forget a rule; soundness can be easier to mess up!

\section{Proof techniques using operational semantics}

A precise language specification lets us precisely prove properties of our
language or programs written in it (and analyses of those programs!).  Note that
this exposition primarily uses big-step semantics to illustrate, but the concepts
generalize. 

\vspace{1ex}
\noindent\textbf{Well-founded induction.}
A key family of proof techniques in programming languages is based on
\emph{induction}.  You may already be familiar with \emph{mathematical
  induction}. As a reminder: if $P(n)$ is a property of the natural numbers that
we want to show holds for all $n$, mathematical induction says that it suffices
to show that $P(0)$ is true (the base case), and then that if $P(m)$ is true,
then so is $P(m + 1)$ for any natural number $m$ (the inductive step). This
works because there are no infinite descending chains of natural numbers.  So,
for any $n$, $P(n)$ can be obtained by simply starting from the base case and
applying $n$ instances of the inductive step.

Mathematical induction is a special case of \emph{well-founded induction}, a
general, powerful proof principle that works as follows: a relation $\prec \ 
\subseteq A \times A$ is well-founded if there are no infinite descending chains
in $A$.  If so, to prove $\forall x \in A. P(x)$ it is enough to prove $\forall
x \in A. [\forall y \prec x \Rightarrow P(y)] \Rightarrow
P(x)$; the base case arises when there is no $y \prec x$, and so the part of the formula within the brackets $[]$ is vacuously true.\footnote{Mathematical induction as a special case arises when $\prec$
  is simply the predecessor relation (${(x, x + 1) | x \in \Natural}$).}

\vspace{1ex}
\noindent\textbf{Structural induction.} 
\emph{Structural induction} is another special case of well-founded induction
where the $\prec$ relation is defined on the structure of a program or a
derivation.  For example, consider the syntax of arithmetic expressions in
\WhileLang, \texttt{Aexp}.  Induction on a recursive definition like this proves a
property about a mathematical structure by demonstrating that the property
holds for all possible forms of that structure.  We define the relation $a \prec
b$ to hold if $a$ is a substructure of $b$.  For \texttt{Aexp} expressions,
the relation $\prec \ \subseteq \mathtt{Aexp} \times \mathtt{Aexp}$ is:

\begin{IEEEeqnarray*}{c}
a_1 \prec a_1 + a_2 \\
a_1 \prec a_1 * a_2 \\
a_2 \prec a_1 + a_2 \\
a_2 \prec a_1 * a_2 \\
\mbox{\ldots \emph{etc., for all arithmetic operators}}~op_a 
\end{IEEEeqnarray*}

To prove that a property $P$ holds for all arithmetic expressions in \WhileLang (or,
$\forall a \in \mathtt{Aexp}.P(a)$), we must show $P$ holds for both the base
cases and the inductive cases.  $a$ is a base case if there is no $a'$ such that
$a' \prec a$; $a$ is an inductive case if $\exists a' ~.~ a' \prec a$.  There
is thus one proof case per form of the expression.  For \texttt{Aexp}, the base
cases are:

\begin{IEEEeqnarray*}{l}
\vdash \forall n \in Z ~.~ P(n) \\
\vdash \forall x \in \mbox{Vars} ~.~ P(x)
\end{IEEEeqnarray*}

\noindent And the inductive cases:
\begin{IEEEeqnarray*}{c}
\vdash \forall a_1, a_2 \in \mathtt{Aexp}\ .\ P(a_1) \land P(a_2) \Rightarrow P(a_1 + a_2) \\
\vdash \forall a_1, a_2 \in \mathtt{Aexp}\ .\ P(a_1) \land P(a_2) \Rightarrow P(a_1 * a_2) \\
\mbox{\ldots \emph{and so on for the other arithmetic operators}\ldots}
\end{IEEEeqnarray*}

\vspace{1ex} \noindent \emph{Example.} Let $L(a)$ be the number of literals and
variable occurrences in some expression $a$ and $O(a)$ be the number of operators
in $a$.  Prove by induction on the structure of $a$ that $\forall a \in
\mathtt{Aexp}\ .\  L(a) = O(a) + 1$:

\vspace{1ex}
\noindent\textbf{Base cases:}
\begin{itemize}[labelwidth=0.7em, labelsep=0.6em, topsep=0ex, itemsep=0ex,
  parsep=0ex] 
\item $\mbox{Case}~ a = n .\  L(a) = 1 ~\mbox{and}~ O(a) = 0$
\item $\mbox{Case}~ a = x .\ L(a) = 1 ~\mbox{and}~ O(a) = 0$
\end{itemize}

\noindent\textbf{Inductive case 1:} $\mbox{Case}~a = a_1 + a_2$ 
\begin{itemize}[labelwidth=0.7em, labelsep=0.6em, topsep=0ex, itemsep=0ex,
  parsep=0ex] 
\item By definition, $L(a) = L(a_1) + L(a_2)$ and $O(a) = O(a_1) + O(a_2) + 1$.
\item By the induction hypothesis, $L(a_1) = O(a_1) + 1$ and $L(a_2) = O(a_2) + 1$.
\item Thus, $L(a) = O(a_1) + O(a_2) + 2 = O(a) + 1.$
\end{itemize}

\vspace{1ex}
\noindent The other arithmetic operators follow the same logic.

\vspace{1ex} Other proofs for the expression sublanguages of \WhileLang can be
similarly conducted.  For example, we could prove that the small-step and
big-step semantics will obtain equivalent results on expressions:

\begin{equation*}
\forall a \in \mathtt{AExp}\ .\ \langle a, E \rangle \rightarrow_a^* n \Leftrightarrow \langle a, E \rangle \Downarrow n
\end{equation*}

The actual proof is left as an exercise, but note that this works because the
semantics rules for expressions are strictly syntax-directed: the meaning of an
expression is determined entirely by the meaning of its subexpressions, the
structure of which guides the induction.

\vspace{1ex}
\noindent\textbf{Induction on the structure of derivations.} Unfortunately, that last
statement is \emph{not} true for \emph{statements} in the \WhileLang language.  For
example, imagine we'd like to prove that \WhileLang is \emph{deterministic} (that
is, if a statement terminates, it always evaluates to the same value).  More
formally, we want to prove that:

\begin{IEEEeqnarray}{lllll}
\forall a \in \mathtt{Aexp}~. ~~ & \forall E ~ .~ \forall n, n' \in \Natural~.~ & ~~~ & \langle a,E \rangle \Downarrow n \wedge \langle a, E \rangle \Downarrow n' \implies n = n' \\
\forall P \in \mathtt{Bexp}~. ~~ & \forall E ~ .~ \forall b, b' \in \mathcal{B}~.~ & ~~~ & \langle P,E \rangle \Downarrow b \wedge \langle P, E \rangle \Downarrow b' \implies b = b' \\
\forall S ~. & \forall E, E',E''~ . & ~~~ & \langle S,E \rangle \Downarrow E' \wedge \langle S, E \rangle \Downarrow E'' \implies E' = E''
\end{IEEEeqnarray}

We can't prove the third statement with structural induction on the language syntax because the
evaluation of statements (like \texttt{while}) does \emph{not} depend only on
the evaluation of its subexpressions.

Fortunately, there is another way.  Recall that the operational semantics assign
meaning to programs by providing rules of inference that allow us to prove
judgements by making derivations. Derivation trees (like the expression trees we
discussed above) are also defined inductively, and are built of
sub-derivations. Because they have structure, we can again use structural
induction, but here, on the structure of derivations.

Instead of assuming (and reasoning about) some statement $S$, we 
instead assume a derivation $D :: \langle S, E \rangle \Downarrow
E'$ and induct on the structure of that derivation (we define $D :: \mbox{Judgement}$ to mean ``D is the derivation that
proves judgement.'' e.g., $D :: \langle x + 1, E \rangle \Downarrow 2$).  That is, to prove that
property $P$ holds for a statement, we will prove that $P$ holds for all
possible derivations of that statement.  Such a proof consists of the following
steps:

\noindent\textbf{Base cases:} show that $P$ holds for each atomic derivation rule with no premises (of the form
$\infer{S}{}$).

\noindent\textbf{Inductive cases:} For each derivation rule of the form

\begin{equation*}
\infer{S}{H_1 ... H_n}
\end{equation*}

\noindent By the induction hypothesis, $P$ holds for $H_i$, where $i = 1\ldots n$. We then
have to prove that the property is preserved by the derivation using the given
rule of inference.

\vspace{1ex}
A key technique for induction on derivations is \emph{inversion.} Because the number
of forms of rules of inference is finite, we can tell which inference rules
might have been used last in the derivation. For example, given $D :: \langle x
:= 55, E_i \rangle \Downarrow E$, we know (by inversion) that the assignment rule of
inference must be the last rule used in $D$ (because no other rules of inference
involve an assignment statement in their concluding judgment). Similarly, if $D
:: \langle \mbox{while}~P~\mbox{do}~S, E_i \rangle \Downarrow E$, then (by inversion) the last
rule used in $D$ was either the \texttt{while-true} rule or the \texttt{while-false} rule.

Given those preliminaries, to prove that the evaluation of statements is
deterministic (equation (3) above), pick arbitrary $S, E, E'$, and $D :: \langle
S, E \rangle \Downarrow E'$

\vspace{1ex}
\noindent \emph{Proof:} by induction of the structure of the derivation $D$, which we define $D :: \langle S, E\rangle \Downarrow E'$. 

\vspace{1ex}
\noindent \textbf{Base case:} the one rule with no premises, \texttt{skip}:

\begin{equation*}
D :: \infer{\langle \mathtt{skip}, E \rangle \Downarrow E}{}
\end{equation*}

By inversion, the last rule used in $D'$ (which, again, produced $E''$) must also
have been the rule for \texttt{skip}.  By the structure of the \texttt{skip}
rule, we know $E'' = E$.

\vspace{1ex} 
\noindent\textbf{Inductive cases:} We need to show that the property holds when
the last rule used in $D$ was each of the possible non-skip \WhileLang commands.  I
will show you one representative case; the rest are left as an exercise.  If the
last rule used was the \texttt{while-true} statement:

\begin{equation*}
D :: 
\infer{\langle \mathtt{while}~P~\mathtt{do}~S, E \rangle \Downarrow E'}
{D_1 :: \langle P, E  \rangle \Downarrow \mathtt{true} & D_2 :: \langle S, E \rangle \Downarrow E_1 &
D_3 :: \langle \mathtt{while}~P~\mathtt{do}~S, E_1 \rangle \Downarrow E'}
\end{equation*}

\noindent Pick arbitrary $E''$ such that $D'' :: \langle \mathtt{while}~P~\mathtt{do}~S, E \rangle \Downarrow E''$

By inversion, and determinism of boolean expressions, $D''$ must also use
the same \texttt{while-true} rule.  So $D''$ must also have subderivations $D_2'' ::
\langle S, E \rangle \Downarrow E_1''$ and $D_3'' :: \langle \mathtt{while}~P~\mathtt{do}~S, E_1 ''
\rangle \Downarrow E''$.  By the induction hypothesis on $D_2$ with $D_2''$, we know
$E_1 = E_1''$.  Using this result and the induction hypothesis on $D_3$ with $D_3''$,
we have $E''= E'$.


\end{document}
