\documentclass[11pt]{article}
\usepackage{../../tex/math-cmds}
\usepackage{../../tex/analysis}
\usepackage{IEEEtrantools}
\usepackage{mathabx}
\usepackage{upgreek}


\title{Lecture Notes: Concolic Testing}
\author{17-355/17-665/17-819: Program Analysis (Spring 2020)\\
        Claire Le Goues\footnote{These notes were developed together with
          Jonathan Aldrich}\\
		{\tt clegoues@cs.cmu.edu}}
\date{}

\begin{document}

\maketitle

%\textbf{TODO TODO 2020 and beyond:} This lecture takes about 50 minutes to
%present.  Add some material or get into another lecture.  
% Note for the future: in 2020 this lecture happened almost immediately after we
% went remote for Coronavirus, so it's not obvious that the time it took to
% cover this material + symexe generalizes to a normal year. 

% Also, something to check out: can SMT solvers figure out that to get
% crypt(x)==crypt(y) you can set x=y, using uninterpreted functions? 

\section{Introduction}

We have discussed symbolic execution from two perspectives: as a method for
forward verification condition generation, as well as a method that generalizes
testing.  We will continue to focus on this latter perspective by discussing key 
approaches that have allowed symbolic execution to find real bugs in practice. 

\subsection{Motivation}

Companies today spend a huge amount of time and energy testing software to
determine whether it does the right thing, and to find and then eliminate bugs.
A major challenge is writing adequate test cases that cover all of the source
code, as well as finding inputs that lead to difficult-to-trigger corner case
defects.

Symbolic execution is a promising approach to exploring different execution
paths through programs. However, it has significant limitations. For paths that
are long and involve many conditions, SMT solvers may not be able to find
satisfying assignments to variables that lead to a test case that follows that
path. Other paths may be short but involve computations that are outside the
capabilities of the solver, such as non-linear arithmetic or cryptographic
functions. For example, consider the following function:

\begin{lstlisting}
testme(int x, int y){
	if(bbox(x)==y){
		ERROR;
	} else {
		// OK
	} 	
}
\end{lstlisting}

If we assume that the implementation of \texttt{bbox} is unavailable, or is too
complicated for a theorem prover to reason about, then symbolic execution may
not be able to determine whether the error is reachable.

%TODO: consider adding more examples from "Existing Approach I/II" slides by Sen to motivate

\subsection{Statically modeling functions}

We have several options for symbolically executing a program with functions
(like the one we developed for interprocedural dataflow analysis).  Inlining is
somewhat more practical here as we are not computing fixpoints.  We can also
simply symbolically execute the called methods, too;  because we are not joining
abstract state over multiple possible paths, we do not immediately lose
precision as we would in interprocedural abstract interpretation.

If we continue to operate in a language with pre and postconditions specified at
the function level (as we assumed in Hoare-Style verification), we can also use
those to model function behavior statically.  Assuming pre- and post-conditions encoded
in the same expression language as guards, $e_{\mathtt{pre}}$ and
$e_{\mathtt{post}}$:

\[
\begin{array}{c}
\infer[\textit{big-return}]{\langle g,\Sigma, \mathtt{return} \rangle \Downarrow \langle a_{\mathtt{post}} ,\Sigma \rangle} {\langle e_{\mathtt{post}},\Sigma \rangle \Downarrow a_{\mathtt{post}}} \\
\end{array}
\]

\noindent \emph{Question: what about function calls? Note that if the language
  involves heap-manipulation, this question becomes more or less difficult!}  

At some point, however, symbolic execution will reach the ``edges'' of the
application: a library, system, or assembly code call. For certain libraries, a
simpler version is available (such as \texttt{libc} implemented for embedded
systems). Other tools allow custom code models, such as the implementation of a
ramdisk to model kernel fs code. This is of course very labor intensive. Even
when this code can be pulled in and executed symbolically, there are times that
the code is simply too complicated to be tractably reasoned about statically,
such as if it involves non-linear arithmetic.

The challenges of fully statically symbolically executing all code directly
motivate \emph{concolic testing}. Concolic testing combines \textbf{conc}rete
execution (i.e. testing) with symb\textbf{olic} execution.\footnote{The word
  concolic is a portmanteau of \textbf{conc}rete and symb\textbf{olic}}

\subsection{Goals}

We will consider the specific goal of automatically unit testing programs to
find assertion violations and run-time errors such as divide by zero. We can
reduce these problems to input generation: given a statement $s$ in program $P$,
compute input $i$ such that $P(i)$ executes $s$.\footnote{This formulation is
  due to Wolfram Schulte} For example, if we have a statement \texttt{assert x >
  5}, we can translate that into the code:

\lstset{
         numbers=left
}

\begin{lstlisting}
if (!(x > 5))
    ERROR;
\end{lstlisting}

Now if line 2 is reachable, the assertion is violated. We can play a similar
trick with run-time errors. For example, a statement involving division
\texttt{x = 3 / i} can be placed under a guard:

\begin{lstlisting}
if (i != 0)
    x = 3 / i;
else
    ERROR;
\end{lstlisting}



\section{Concolic execution overview}

In concolic execution, symbolic execution is used to solve for inputs that lead
along a certain path. However, when a part of the path condition is infeasible
for the SMT solver to handle, we substitute values from a test run of the
program. In many cases, this allows us to make progress towards covering parts
of the code that we could not reach through either symbolic execution or
randomly generated tests. 

Consider the \texttt{testme} example from the motivating section. Although
symbolic analysis cannot solve for values of \texttt{x} and \texttt{y} that
allow execution to reach the error, we can generate random test cases. These
random test cases are unlikely to reach the error: for each \texttt{x} there is
only one \texttt{y} that will work, and random input generation is unlikely to
find it. However, concolic testing can use the concrete value of \texttt{x} and
the result of running \texttt{bbox(x)} in order to solve for a matching
\texttt{y} value. Running the code with the original \texttt{x} and the solution
for \texttt{y} results in a test case that reaches the error.

In order to understand how concolic testing works in detail, consider a more
realistic and more complete example:

\begin{lstlisting}
int double (int v) { 
	return 2*v; 
}

void bar(int x, int y) {
	z = double (y);
	if (z == x) {
		if (x > y+10) {
		      ERROR;
		}	
	}
}
\end{lstlisting}

We want to test the function \texttt{bar}. We start with random inputs such as
$x=22, y=7$. We then run the test case and look at the path that is taken by
execution: in this case, we compute $z=14$ and skip the outer conditional. We
then execute symbolically along this path. Given inputs $x=x_0, y=y_0$, we
discover that at the end of execution $z=2*y_0$, and we come up with a path
condition $2*y_0 \neq x_0$.

In order to reach other statements in the program, the concolic execution engine
picks a branch to reverse. In this case there is only one branch touched by the
current execution path; this is the branch that produced the path condition
above. We negate the path condition to get $2*y_0 == x_0$ and ask the SMT solver
to give us a satisfying solution.

Assume the SMT solver produces the solution $x_0=2, y_0=1$. We run the code with
that input. This time the first branch is taken but the second one is not.
Symbolic execution returns the same end result, but this time produces a path
condition $2*y_0 == x_0 \land x_0 \leq y_0+10$.

Now to explore a different path we could reverse either test, but we've already
explored the path that involves negating the first condition. So in order to
explore new code, the concolic execution engine negates the condition from the
second \texttt{if} statement, leaving the first as-is. We hand the formula
$2*y_0 == x_0 \land x_0 > y_0+10$ to an SMT solver, which produces a solution
$x_0=30, y_0=15$. This input leads to the error.

The example above involves no problematic SMT formulas, so regular symbolic
execution would suffice. The following example illustrates a variant of the
example in which concolic execution is essential:

\begin{lstlisting}
int foo(int v) { 
	return v*v%50;
}

void baz(int x, int y) {
	z = foo(y);
	if (z == x) {
		if (x > y+10) {
		      ERROR;
		}	
	}
}
\end{lstlisting}

Although the code to be tested in \texttt{baz} is almost the same as \texttt{bar} above, the problem is more difficult because of the non-linear arithmetic and the modulus operator in \texttt{foo}.  If we take the same two initial inputs, $x=22, y=7$, symbolic execution gives us the formula $z=(y_0*y_0)\%50$, and the path condition is $x_0 \neq (y_0*y_0)\%50$.  This formula is not linear in the input $y_0$, and so it may defeat the SMT solver.

We can address the issue by treating \texttt{foo}, the function that includes nonlinear computation, concretely instead of symbolically.  In the symbolic state we now get $z=foo(y_0)$, and for $y_0=7$ we have $z=49$.  The path condition becaomse $foo(y_0) \neq x_0$, and when we negate this we get $foo(y_0)==x_0$, or $49==x_0$.  This is trivially solvable with $x_0==49$.  We leave $y_0=7$ as before; this is the best choice because $y_0$ is an input to $foo(y_0)$ so if we change it, then setting $x_0=49$ may not lead to taking the first conditional.  In this case, the new test case of $x=49, y=7$ finds the error.


\section{Implementation}

Ball and Daniel~\cite{Ball2015} give the following pseudocode for concolic execution (which they call dynamic symbolic execution):

\begin{lstlisting}
i = an input to program P
while defined(i):
   p = path covered by execution P(i)
   cond = pathCondition(p)
   s = SMT(Not(cond))
   i = s.model()
\end{lstlisting}

Broadly, this just systematizes the approach illustrated in the previous section.  However, a number of details are worth noting:

First, when negating the path condition, there is a choice about how to do it.  As discussed above, the usual approach is to put the path conditions in the order in which they were generated by symbolic execution.  The concolic execution engine may target a particular region of code for execution.  It finds the first branch for which the path to that region diverges from the current test case.  The path conditions are left unchanged up to this branch, but the condition for this branch is negated.  Any conditions beyond the branch under consideration are simply omitted.  With this approach, the solution provided by the SMT solver will result in execution reaching the branch and then taking it in the opposite direction, leading execution closer to the targeted region of code.

Second, when generating the path condition, the concolic execution engine may choose to replace some expressions with constants taken from the run of the test case, rather than treating those expressions symbolically.  These expressions can be chosen for one of several reasons.  First, we may choose formulas that are difficult to invert, such as non-linear arithmetic or cryptographic hash functions.  Second, we may choose code that is highly complex, leading to formulas that are too large to solve efficiently.  Third, we may decide that some code is not important to test, such as low-level libraries that the code we are writing depends on.  While sometimes these libraries could be analyzable, when they add no value to the testing process, they simply make the formulas harder to solve than they are when the libraries are analyzed using concrete data.



%        for soundness, the inputs those expressions depend on should also be fixed
%            could be hard to compute if there is a complicated expression we aren't analyzing
%            can give up soundness; often works hueristically anyway, if the input variables are not otherwise relevant they won't be in the formula and we'll use the old values next time by default


\section{Concolic Path Condition Soundness}

Concolic execution is motivated by the presence of subexpressions within a path
condition that are difficult for a SMT solver to reason about. The key idea of
concolic execution is to replace these subexpressions with appropriate concrete
values.
%
Where possible, we would like this replacement to be \emph{sound}. Intuitively,
a replacement is sound if any solution to the new path condition is also a
solution to the old one. This means that even after the substitution, concolic
execution will successfully drive the program down the desired path. Let's make
this idea more formal.

Let $g$ be a negated path condition. Let $M$ be a map from symbolic constants
$\alpha$ to integers $n$. We write $[M]g$ for the boolean expression we get by
substituting all the symbolic constants in $g$ with the corresponding integer
values given in $M$; this is only defined if the free symbolic constants $FC(g)$
are the same as $\textit{domain}(M)$. We define $[M]a_s$ similarly for
substitution of symbolic constants with values in arithmetic expressions.

Given $g$ and a map $M$ that represents the inputs to a concrete test case
execution, concolic execution may replace a subexpression $a_s$ of $g$ with the
concrete value $n$ achieved in testing. Note that $n=[M]a_s$. Let the new guard
be $g'=[n/a_s]g$ (again, we consider this \textit{after} negating the last
constraint in the path).

We say that $g'$ is a \textit{sound} concolic path condition if for all
alternative test inputs $M'$ such that $[M']g'$ is true, we have
$[extend(M',M)]g$ true. Here, the $extend$ function extends the symbolic
constants in $M'$ with any that are necessary to match the domain of $M$. More
precisely, $\forall \alpha' \in domain(M'), extend(M',M)[\alpha']=M'[\alpha']$
and $\forall \alpha \in (domain(M)-domain(M')), extend(M',M)[\alpha]=M[\alpha]$.

In class we saw an example of a path condition $g$ and a sound concolic
replacement $g'$ for it. In particular, $g$ was $x_0 == (y_0*y_0)\%50$ after
negation and $g'$ was $x_0==49$ after negation. This is trivially sound because
the only solution is $x_0==49$, which when extended with $y_0==7$ from the
original test case yields a new test input that fulfills the original path
condition $x_0 == (y_0*y_0)\%50$.

As an exercise:
\begin{itemize}
\item Give an example path condition $g$, test input $M$, and concolic path
  condition $g'$ resulting from replacing a subexpression $a_s$ of $g$ with a
  concrete value $n=[M]a_s$, such that $g'$ is \textit{unsound}.

\item Witness the unsoundness by also providing a test input $M'$ that satisfies
  $g'$ but not $g$.

\item Give a condition on $g, M, g'$ and/or $a_s$ that is sufficient to ensure that $g'$ is sound.
\item Prove that your condition is sufficient for soundness.
\end{itemize}

\section{Acknowledgments}

The structure of these notes and the examples are adapted from a presentation by Koushik Sen.

\bibliographystyle{abbrv}
\bibliography{symbolic}

\end{document}
